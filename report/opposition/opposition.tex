\documentclass{article}
\title{Opposition}
\author{Oskar Ingemarsson \& Sebastion Weddmark Olsson}
\begin{document}
\date{\today}
\maketitle

\section*{Good Stuff}
\begin{itemize}
\item The figures are elegant, well explained and makes it much easier to
understand the report.
\item From the report it becomes clear that you have achieved great results and
greatly approved the running time compared with a sequential lexical analyser.
\item The discussion chapter has great potential.
\end{itemize}
\section*{Remarks and Tips}
\begin{itemize}
\item The introduction chapter is very short. A general explanation for what you
intend to accomplish and what limitations are would be nice. Its also worth
mention to explain more in detail why your work are relevant and how it can
contribute.

\item It is possible to add shorter sections about trivial stuff like what
is Haskell, what is Alex, why is an lexical analysers needed, etcetera? Dividing
up such things in shorter section makes it more distinct what you intend to do
and what tools and concepts you will use. Preferable such things could be put in
a theory chapter.

\item Why was Alex and Haskell chosen? Very little about this is mentioned in
discussion. Why first in the discussion and why in the discussion at all? You
only say that a lot of work was already done in Haskell but mention both Lex and
Flex as alternatives to Alex. A short comprehension between those tools would be
nice. If you want to mention other tools for generating lexical analyzers then
Alex in the discussion, it should be about what problems arose when using Alex
and which problems could be avoided using other tools etcetera.  Not why you
chose Alex and Haskell, that belongs in a method chapter!
\item What kind of facts are mentioned in the chapter \emph{Lexer} and
\emph{Divide and Conquer Lexer}? Are they strictly theory or have you done
your own conclusions as well? If this is strictly theory than there are
references missing.
\item Maybe you should follow the standard outline proposed by
Chalmers. It is hard to follow where and when you are introducing concepts and
where you are make your own conclusions.
\item For what language have you been writing a lexical analyser? There are an
appendix about Java Lette Light but it is not clear that this is the language
that you have used in the report. Except from the appendix Java Lette Light are
never mentioned but the programming language C are mentioned several times.
\item Think about how you introduce new concept and make sure that they are
explained before they are used. For instance
\emph{monoid} is mentioned and used several times before it is explained in
detail. Again think of the structure of the report.
\item It should be clarified what you are benchmarking in the result. It becomes
clear first in the discussion that the sequential lexical analyzer is generated by Alex.
The discussion chapter should discuss the results that you already presented.
You should not add more result, concepts and tools here; they should already
have been explained.
\end{itemize}

\section{Questions}
\begin{itemize}
\item Why was finger trees chosen? Have you considered other data structures as
well?
\item Is it possible to improve auto completion using your lexical analyser?
\item Since it takes a lot of time building the incremental lexing tree from
scratch, have you considered saving the tree to a file when closing the
application where the lexing occurs? In other words using some kind of
serialization.
\end{itemize}

\section{Conclusion}
The main problem seems to be structure. Fixing this among other minor things
this will make an excellent report.

\end{document}
