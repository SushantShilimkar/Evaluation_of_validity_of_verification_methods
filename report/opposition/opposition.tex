\documentclass{article}
\title{Opposition}
\author{Oskar \& Zebrastian}
\begin{document}
\date{\today}
\maketitle

\section*{Good Stuff}
$\emptyset$
\section*{Bad Stuff}
\begin{itemize}
\item Why was Alex and Haskell chosen? Very little about this is mentioned in
discussion. Why first in the discussion and why in the discussion at all? You
only say that a lot of work was already done in Haskell but mention both Lex and
Flex. A short comprehension between those lexers would be nice. If you want to
mention other lexers then Alex in the discussion, it should be about
what problems arose when using Alex and which problems could be avoided using
other lexers etcetera.
Not why you chose Alex and Haskell, that belongs in a method chapter!
\item What kind of facts are mentioned in the chapter \emph{Lexer} and
\emph{Divide and Conquer Lexer}? Are they strictly theory or have you done
your own conclusions as well? If this is strictly theory than there are
references missing.
\item Maybe you should follow the standard outline proposed by
Chalmers. It is hard to follow where and when you are introducing concepts and
where you are make your own conclusions.
\item What language are you actually lexing? There are an appendix abut Java
Lette Light but it is not clear that this is the language that you have used in
the report. Except from the appendix Java Lette Light are never mentioned but
the programming language C are mentioned several times.
\item Think about how you introduce new concept and make sure that they are
explained before they are used. For instance
\emph{monoid} is mentioned and used several times before it is explained in
detail. Again think of the structure of the report.
\item Why was finger trees chosen? Have you considered other data structures as
\end{itemize}
\end{document}
