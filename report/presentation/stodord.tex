\documentclass[a4paper]{article}
\usepackage[utf8]{inputenc}
\usepackage[T1]{fontenc}

\begin{document}
\section{Titlepage}
My name is X and this is Y. We will present our master thesis
``Evaluation of validity of verification methods'' that was done at
Mecel AB.

It is about automating testing of software modules in the automotive
industry.

\section{Table of contents}

\section{Background}
Vehicles become more and more complex. Today there are around 80
electronic control units, running millions of lines of code needed for
brakes, infotainment systems, steering, windscreen wipers and so on.

More functionality is added every year. Things like automated parking,
night vision, traffic sign recognision will be mandatory in new cars
in the future. Cars will be connected with each other, they will be
connected with roadside management systems and also with the Internet.

Vehicles are already safety critical, because you don't want to crash
because of a software failure.

Already different components share a limited number of
resources. Adding more functionality means even more components will
share resources. For example, you don't want your brakes to stop
working because your stereo has a software malfunction.

Research has shown that testing accounts for around half of all
software development costs.

\section{Glossary-1}
Before we move on we have some key words you will need to comprehend.

AUTOSAR is a standard for implementing vehicle software components,
which consists of around 80 module specifications.

AUTOSAR's concept is to make it possible for vehicle manufacturers to
buy components from different software companies, which will still
work together in unison.

\section{Glossary-2}
Functional safety is a relative new concept in the automotive
industry.

It defines the development process from idea to product.

Every step in the process must fulfill certain requirements.

The main idea is if a component in the system fails, then the whole
system should not fail.

\section{Glossary-3}
ASIL refers to the classification of functional safety.

A software function with a higher ASIL classification need to
experience more rigorous testing.

\section{Objectives}
Our objectives are to automate the testing of AUTOSAR modules. The
goal is to reduce the costs of testing and make the code more
reliable.

We also aim to examine if it is possible to reach a higher ASIL classification.

\section{Watchdog manager}
Since AUTOSAR consists of many module specifications, we had to start
somewhere and choose one specification.

We choose the watchdog manager.

The watchdog is a hardware component that is used to detect and
recover from computer malfunctions.

The watchdog manager configures and supervises all watchdogs in the
system.

\textbf{CONFIGURATION}

\section{Glossary-4}
A supervised entity is a critical section in a piece of code that
needs to follow a certain behavior.

\textbf{CHECKPOINTS}

\section{Functions in WdgM}
AUTOSAR specifies eleven API calls for the watchdog manager.

These can be divided into two groups; get-functions, and functions
that changes the state of the watchdog manager.

The get functions shall of course follow certain requirements, but the
will not change the state since their only task is to retrieve
information about the state.

Then there is the set functions that changes the state of the Watchdog
manager.

The Init function must be called first, because it initializes the
watchdog manager. It takes a configuration as argument.

the DeInit function is the opposite of Init; it deactivates the
watchdog manager.

PerformReset causes an external hardware reset on all configured
hardware watchdogs.

SetMode changes the mode of the watchdog manager. Different things can
be configured in different modes.

The most interesting things happen in the MainFunction and
CheckpointReached. These two functions handles the supervision mechanisms.
The MainFunction is called in an interval by the run time environment.
It switches the global status.

CheckpointReached is called when the system reaches a critical point a
so called checkpoint.

\section{Supervision mechanisms}
To maintain supervision of components there are three different
mechanisms used by the Watchdog manager.

Alive supervision verifies that checkpoints is called periodically.

Deadline supervision verifies that checkpoints is called within a time
limit.

Logical supervision verifies that checkpoints is reached in certain sequences.

\section{The state machine}
The behavior of the watchdog manager depends on its global status
which is calculated from the statuses of the supervised
entities. Those statuses is called local statuses.

A supervised entity's local status is calculated with the help of the
supervision mechanisms mentioned.

\section{Global statuses}
The state machine begins with the global status deactivated (as seen
in this picture).

When the initialization function is called the global status changes
to OK.
It will stay in OK as long as the behavior of the watchdog manager,
according to the local statuses, is correct.

In case of a failure, the global status changes to failed, expired or
stopped depending on the configuration and the failure.

The watchdog manager can recover when the global status is failed.

The global status stopped is an absorbing state and the Watchdog
manager can not recover from this state by itself. What is needed is
an external hardware reset. When in the absorbing state no supervision
mechanisms is executed.

The expired state is a way for the watchdog manager to postpone error
reaction.

\section{Generating C byte code}
The watchdog manager can be configured in a number of different
ways. Therefor the configuration must be supplied before the C code
can be compiled. Dependencies to other modules must also be included.

\section{Glossary-5}
We have used QuickCheck for automating tests. QuickCheck is a model
based testing tool which generates arbitrary sequences of API calls.

\section{QuickCheck}
Because QuickCheck is model based, AUTOSAR specifications first need
to be translated into a model written the Erlang programming language,
thus readable by QuickCheck.

QuickCheck calls each generated API call and compares the result from
the C code with the result from the C code with the result from the
Erlang code.

\section{Configurations}
As we said before, the watchdog manager can be configured in different
ways. Therefor we tested the watchdog manager using three different
configurations. Using different configurations make the testing more
reliable.

These three configurations are called the BSI, the Example and the
Freescale configuration.

The BSI configuration is a highly simplified configuration. For
example it has no supervision mechanisms configured, which make it
impossible to reach an incorrect global state.
Because of this, everything cant be tested using this configuration.

The example configuration is very complex, which means a lot of things
can be tested. It also means that a lot of things can cause an invalid
behavior of the Watchdog manager and put the watchdog manager in an
absorbing failure state.

The Freescale configuration is a realistic configuration, which is
used in lab environments at Mecel. Using this configuration, all
transitions in the state machine was reached.

\section{Testing}
We have implemented both positive and negative testing, where positive
testing tries to only generate correct command sequences with correct
arguments. The Watchdog manager is supposed to stay in the global
status OK or DEACTIVATED.

The requirements of the Watchdog manager must hold even if command
sequences or arguments to API calls are invalid to achieve functional
safety. Negative testing tries to ensure this. We have performed this
by passing null pointers and other invalid arguments.

The con with negative testing is that it is very easy to get to an
absorbing state. All commands that is generated after reaching an
absorbing state will not test anything.

\section{Bug handling}
When encountering a bug there were a question that arose. How should
we continue?

A problem with QuickCheck is that it generates the command sequences
before the actual execution of tests. This make it impossible to
``skip'' a bug and continue to search for others.

The best way would be to report the bug, and let a third party fix
it. However this may be time consuming and it will probably take weeks
before the bug is patched and we can continue testing.

Letting a third party fix the bug will ensure that they agree on that
this actually is a bug. They will also be able to fix the bug
independently of our Erlang implementation.

Another method is to change the Erlang model to follow the C code,
hence making the bug to appear in Erlang as well.

Doing this we wont be able to find secondary failures, or use
different configurations or other versions of the module.

Fixing the C code ourselves makes it easy to adapt the same techniques
that we used in the Erlang implementation and translate them into C
code.
This increases the risk of ending up with the same bugs in both C code
and Erlang. Such bugs will not be discovered by QuickCheck since both
implementations will agree on the same incorrect model.

\section{Bugs we found}
A lot of bugs were found in the C-code, even bugs that was considered safety
critical.  Bugs that were considered safety critical were mainly bugs in the
supervision algorithms.  Most of the bugs were easy to fix but some needed a
lot of C code to be rewritten.

Even bugs in the generated code were found. Such bugs corrupted mainly the
initial state of the watchdog manager.

A lot of errors and ambiguity's etcetera were found in the AUTOSAR
specification. Errors like misspelling of configuration parameter etcetera are
not very likely to cause errors in implementations but still doesn't make the
standard very trustworthy. Worst are however ambiguities in the specification,
there were situations when we could not tell who was faulty. Our Erlang
implementation or the C code? In those cases we choose to follow the C
implementation.

\section{Coverage \& Statistics}
The reason we don't achieve 100\% code coverage depends on certain limitations.
Some lines can not be covered when the configuration parameter
\emph{WdgMDevErrorDetect} is true. On the other hand, if it is
false, then the C model will fail with a segmentation fault and the
Erlang model will not be covered any way.

The missing coverage in the C-code are for example checks for null
pointers, some of which never evaluated to false. Many checks seems to be
redundant and impossible to evaluate to true, if one excludes the possibility
of hardware failures or other failures which may corrupt the memory of the
watchdog manager.

\section{Functional safety}
The configuration holds a dangerous parameter that is called
DevErrorDetect. This parameter will allow the Watchdog manager to
check for development errors like null pointers. If the parameter is
turned of and a function is called with invalid data, then the whole
system will crash.

This parameter is actually turned off before the code is sent to production.

\section{Future work}
If several AUTOSAR modules is translated into Erlang code, they can be
combined and QuickCheck can then test the dependencies of the whole
connected system.

One can also try to test more configurations in order to gain even
better test results.

\section{Conclusions}
QuickCheck can be used in some extent to implement functional
safety. The problem is that AUTOSAR is written with informal syntax
and therefor must first be translated into formal syntax. Also
functional safety is a process that stretches from idea to product,
and not only at the software development level.

One needs to be aware that it may be difficult to reach the whole
state space if the configuration is to complex or not complex
enough. There may also be configuration parameters that is not safe
for the reliability of the system to turn off.

It is important to measure the tested state space, but it is equally
important to measure the code coverage. These two are semi independent
of each other. One can have almost 100\% code coverage without covering
the full state space.

\section{Questions?}

\end{document}
