\documentclass[a4paper]{article}
\usepackage[utf8]{inputenc}
\usepackage[T1]{fontenc}

\begin{document}
\section{Titlepage}
My name is X and this is Y. We will present our master thesis
``Evaluation of validity of verification methods'' that was done at
Mecel AB.

It is about automating testing of software modules in the automotive
industry.

\section{Table of contents}

\section{Background}
Vehicles become more and more complex. Today there are around 80
electronic control units, running millions of lines of code needed for
brakes, infotainment systems, steering, windscreen wipers and so on.

More functionality is added every year. Things like automated parking,
night vision, traffic sign recognision will be mandatory in new cars
in the future. Cars will be connected with each other, they will be
connected with roadside management systems and also with the Internet.

Vehicles are already safety critical, because you don't want to crash
because of a software failure.

Already different components share a limited number of
resources. Adding more functionality means even more components will
share resources. For example, you don't want your brakes to stop
working because your stereo has a software malfunction.

Research has shown that testing accounts for around half of all
software development costs.

\section{Glossary-1}
Before we move on we have some key words you will need to comprehend.

AUTOSAR is a standard for implementing vehicle software components,
which consists of around 80 module specifications.

AUTOSAR's concept is to make it possible for vehicle manufacturers to
buy components from different software companies, which will still
work together in unison.

\section{Glossary-2}
Functional safety is a relative new concept in the automotive
industry.

It defines the development process from idea to product.

Every step in the process must fulfill certain requirements.

The main idea is if a component in the system fails, then the whole
system should not fail.

\section{Glossary-3}
ASIL refers to the classification of functional safety.

A software function with a higher ASIL classification need to
experience more rigorous testing.

\section{Objectives}
Our objectives are to automate the testing of AUTOSAR modules. The
goal is to reduce the costs of testing and make the code more
reliable.

We also aim to examine if it is possible to reach a higher ASIL classification.

\section{Watchdog manager}
Since AUTOSAR consists of many module specifications, we had to start
somewhere and choose one specification.

We choose the watchdog manager.

The watchdog is a hardware component that is used to detect and
recover from computer malfunctions.

The watchdog manager configures and supervises all watchdogs in the
system.

\textbf{CONFIGURATION}

\section{Glossary-4}
A supervised entity is a critical section in a piece of code that
needs to follow a certain behavior.

\textbf{CHECKPOINTS}

\section{Functions in WdgM}
AUTOSAR specifies eleven API calls for the watchdog manager.

These can be divided into two groups; get-functions, and functions
that changes the state of the watchdog manager.

The get functions shall of course follow certain requirements, but the
will not change the state since their only task is to retrieve
information about the state.

Then there is the set functions that changes the state of the Watchdog
manager.

The init function must be called first, because it initializes the
watchdog manager. It takes a configuration as argument.
\section{Supervision mechanisms}
\section{The state machine}
\section{Global statuses}
\section{Generating C byte code}
\section{Glossary-5}
\section{QuickCheck}
\section{Configurations}
\section{Testing}
\section{Less good bug methods}
\section{Better bug methods}
\section{}
\section{}
\section{}
\section{}
\section{}
\section{}
\section{}
\section{}
\section{}
\section{}


\end{document}