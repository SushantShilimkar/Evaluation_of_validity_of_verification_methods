\section{Verification}
The exact meaning of verification is confusing \cite{thomas_arts}. The
definition may differ in comparison of academic or industrial sense. Even in
different phases of the safety lifecycle verification is conducted in various
forms \cite[Chapter 9.2, Part 8]{ISO26262}.

\subsection{Formal Verification}
%Skriv n�got luddigt
\subsection{Semi foramal verification}
%Skriv n�got �n mer luddigt

\section{Industrial Safety Standards} It is useful to distinguish between
systems with different levels of dependability and determine where the hazards
exists\cite{COURSEBOOK:safety-critical}. When this risk analysis is done and
appropriate reliability and availability requirements is assigned to the
system, the system can be identified by a certain safety integrity level (SIL).
If this number is high, the system will experience a more rigorous design and
testing than could be justified for a lesser demanding system. These levels is
more defined in the standards IEC~61508 and ISO~26262.

%\setion{Functional Safety}
% Kanske n�got vi vill skriva om? N�ms i metoden

\subsection {IEC~61508 and ISO~26262}

%G�r skillnad p� verifikation och verifikation. Dvs vilken fas handlar det om?
Automotive software for safety related systems is required to be designed,
implemented and verified by the standard ISO~26262 that handles functional
safety for automotive equipment.  For a higher level of integrity ISO~26262
strongly recommends that a semi-formal verification of each module should exist
\cite[Part 6]{ISO26262}.  It also recommends a formal verification, but because
of the state-space complexity this is hard to achieve.

ISO~26262 is built on IEC~61508 which is titled Functional Safety of
Electrical/Electronic/Programmable Electronic Safety-related Systems which can
be applied to any kind of industry. IEC~61508 have four safety integrity levels
(SIL) ranked 1-4. SIL 4 is the highest and should be applied where a failure can
do devastating damage to a large area. The automotive industry is improbable to
have this risk. That is why ISO~26262 exists, it also has four levels of SIL
called automotive safety integrity level (ASIL). The ASIL range from A-D, where
D is the highest and roughly translated to SIL 3.

\subsection{AUTOSAR (AUTomotive Open System ARchitecture)}
%Beskriv vad fan det �r
%K�lla http://www.autosar.org/
AUTOSAR is a worldwide development partnership of car manufacturers, suppliers
and other companies from the electronics, semiconductor and software industry.

Some of the specifications in AUTOSAR is left quite open for interpretation.
This makes it possible for vehicle developers to have different specifications
for a configuration. Some parts of those configuration specifications is
generated into code, while other is manually written or added as a configurable.

%\subsection{Automotive Safety Integrity Level (ASIL)}

%ISO 26262 is intended to be applied to safety rated systems that include on or
%more electrical and/or electronic (E/E) systems and that are installed in series
%production passenger cars.

\section{Existing Software}
There are a number of existing software tools whose aim are to simplify, for a
developer, the implementation of verification and testing.
\begin{itemize}
\item QuickCheck
\item SPIN
\item McErlang
\item $\mu$CRL toolset
\item CADP
\item Parasoft C/C++test
%\item Erlang and QuickCheck
%\item Symlink
%\item CORE
%\item Isabelle

\end{itemize}

\subsection{QuickCheck}
% QuickCheck tests a program with a specification implemented as properties
% that the program must hold \cite{QUICKCHECK:manual}. QuickCheck creates an
% arbitrary test vector for each of the properties. Because it is arbitrary it
% can't be used for true formal verification(?).

QuickCheck tests a program with specification implemented as properties that
the program must hold \cite{QUICKCHECK:manual}. QuickCheck has guided random
test generation. This means that the samples can be weighted to cover certain
parts of the state-space with more likelihood.

\subsection{Erlang}
Erlang can easily communicate with other programming language by
using byte streams. There have been some work done including Erlang,
AUTOSAR and QuickCheck, mostly by the QuviQ company\cite{QUVIQ:flyer}.
Imperative coding requires state based testing(?). There are a library in
Erlang developed by QuviQ for this purpose(?).

% \subsection{CORE}
% A specification package developed by British Aerospace and System Designers.

%\subsection{Promela}

\subsection{SPIN}
SPIN is used to trace logical design errors in distributed
software\cite{SPIN:manual}. It supports a high level language, Promela, to
specify system descriptions. Promela stands for Process Meta Language and is a
verification model language. The system properties that should be checked are
written in logical temporal language (LTL) and SPIN reports errors such as
deadlocks, race conditions and incompleteness between these properties and the
system model. It also supports embedded C code as part of the model
specifications. Even further it supports random, interactive and guided
simulation, with both partial or exhaustive proof techniques.

\subsection{McErlang}
McErlang is a model checker written in Erlang used for verifying distributed
Erlang programs\cite{MCERLANG:model-checker}. Its purpose is to check a program
against a correctness property, but can also among other things check safety
or liveness properties.

McErlang offers two depth-first state traversal model checking algorithms, one
checks safety properties and the other is used for checking the liveness.
McErlang also implements weak process fairness, by omitting non-fair loops from the
accepting runs in its liveness algorithm.

\subsection{$\mu$CRL toolset} $\mu$CRL is a process algebraic language which is
suited for the analysis of distributed systems. The toolkit is built on this
language and contains a theorem prover based on binary decision
diagrams\cite{MCRL:manual}.

\subsection{CADP} Construction and Analysis of Distributed Processes (CADP) is
a toolset for design of distributed systems\cite{CADP:manual}. It includes,
among others, tools for equivalence checking, state space manipulation and
model checking and also includes several verification algorithms. It provides
functionality as step-by-step simulation to parallel model-checking.

\subsection{Parasoft C++test}
Parasoft C++ is a commercial software with a huge range of functionality
developed by the Parasoft company with the purpose of testing software written
in C or C++. Parasoft theme self claims that it should be possible to satisfy
ASIL requirements using their software.
%\subsection{Isabelle}
% Isabelle theorem prover is an interactive theorem prover, successor of the
% Higher Order Logic (HOL) theorem prover

\section{Verification Methods}
The standard 61508-7 � IEC:2000 propose to two methods to say that a program is
formal verified. The key is to model the program into one of the following
machines.
\begin{enumerate}
\item \label{enum:FSM} Finite state machines/state transition diagrams
\item Time Petri nets
\end{enumerate}

The standard emphasises that Time Petri nets are best suited for concurrent
programs. Regarding method \ref{enum:FSM} the following criteria needs to be
satisfied for the implemented state machine to be formal verified.

\begin{enumerate}
\item completeness (the system must have an action and new state for every input
in every state);
\item consistency (only one state change is described for each state/input
pair); and
\item reachability (whether or not it is possible to get from one state to
another by any sequence of inputs).
\end{enumerate}

If the state machine is correct implemented, hence is a correct model of the
original program, this also says that the original program is formal verified.

Since most program specification are written in natural languages there may be
lots of ambiguities. Therefor a lots of techniques has been developed to reduce
such cases. This techniques are often referred as semi formal verification,
because they lack of mathematical rigour associated with formal
verification. This methods use textual, graphical or other notation; often
several techniques are used. % These tests may not be

The description of semi formal verification in the 61508-7 � IEC:2000 standard
states: \\
''Semi-formal methods provide a means of developing a description of a system at
some stage in its development, i.e. specification, design or coding. The
description can in some cases be analysed by machine or animated to display
various aspects of the system behaviour.''
