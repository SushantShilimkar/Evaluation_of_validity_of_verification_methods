
The model has been implemented in an iterative way. Function for function,
requirement for requirement. This process is very easy with the use of
QuickCheck.
In the beginning there were alot of negative testing, because we didn't care
about tweaking the generators.
Sometimes the model needs to be corrected because of ambiguities in AUTOSAR or
errors in the model. Quite often it was the C-code that had the errors and
needed to be corrected.
Easier when you do white box testing, because then you can really check the code
and compare with the requirements.

We achieved fair coverage of the c-code, around 90\%, and the Erlang code,
around \%. The problem was the requirements where we achieved around 50\%. It
would help if an QuickCheck model was implemented for the whole system as
well. Many of the requirements had dependencies in other modules, and some
requirements for the file structure, the configuration or even the generation of
files.

QuickCheck is good for overall testing, and can help with raising the functional
safety of modules.

%% analys av resultat
%% - coverage på modell och c-kod genom cover (erlang) och bullseye (c-kod)
%% - statistik på quickcheck körningar.

Stripped of comments and blank lines, the implemented Erlang model is
almost 1500 lines of code. This is to be compared with the c-code
which is over 12500 lines of code.

\section{Future work}
Some interesting thing we wanted to do from the beginning was to implementing
another module, preferably one that has some kind of dependency with the
watchdog manager. This is because then, it would be possible to testing towards
the phase ``system integration and testing'' in ISO~26262 and get even better
results.

Another good idea is to do more negative testing, and testing of null
pointers. This should be done to raise the coverage for the C-code to even
better levels.

We could also try new configurations. More configurations = better testing.