\section{Background}
\subsection{The Development within automotive industry}
In the recent decades there has been a dramatic growth of information
and communication innovations within the automotive industry.  Analog
vehicles have been transformed into complex electromechanical
systems. New features are implemented (for example due to user
demands, traffic safety or environmental regulations) requiring more
computational power and less energy consumption.  The average car has
already 80 ECU's (electronic computation units) and to deal with the
extra functionality, each ECU will need to become more complex.
\cite{CRC:embedded_handbook}
\cite{Cambridge:controlsystems}
\cite{Springer:advanced_microsystems}
\cite{Springer:advanced_microsystems}
\cite{ISO26262}
\cite{QUVIQ:COURSE}

\subsection{Extent of software in modern vehicles}
%% TODO: F�rsk�na f�rsta stycket
Developing a new car model costs up to one billion \euro, where
electronics has reached a mean share of one third of the value,
divided 20\% sensor value, 40\% hardware value and 40\% software
value. The share of software has been doubled the last 10 years.
\cite{Wiley:internetworking}
\cite{Clemens:tech_acceptance}

More and more functions will be implemented; Intelligent traffic
systems which make the automotive vehicles communicate with the
roadside management systems, infotainment systems will bring, among
other, weather information through the Internet and emergency call
support, traffic sign recognition, night vision and automated parking.
\cite{Springer:advanced_microsystems}
\cite{VOLVO:convoys}
\cite{Cambridge:controlsystems}
\cite{Wiley:internetworking}

The number of lines of code running in a vehicle is another example of how
complex the automotive software is. The software running on a F-22 Raptor and
the F-35 Strike Fighter, two of the attack planes in the US air force, has
about 1.7 million lines of code and 5.7 million lines of code respectively. The
passenger plane Boeing 787 Dreamliner runs on 6.5 million lines of code, where
the average premium-class car has close to 100 million lines of code.
\cite{IEEESpectrum:car_code}

\subsection{Introduction of standards}
Because of the high development costs, and the complexity of modern
cars, car manufacturers, suppliers and other companies related to the
automotive industry joined efforts in 2003 and created AUTOSAR, short
for Automotive Open System Architecture. The
main purpose is to make it possible for car manufacturers to buy
independent components from different software suppliers; the AUTOSAR
motto is ``Cooperate on standards, compete on implementation''.
\cite{AUTOSAR:basic_info}

%% TODO: �verv�g att byta "ISO 26262" i andra meningen mot n�got mer givande.
Functional safety was introduced to the automotive industry with ISO
26262 in late 2011. ISO 26262 named ``Road vehicles --
Functional safety'' is a general standard on how the implementation of
functional safety in vehicle development should be carried out from
beginning to end.
\cite{ISO26262}

This standard is built on top of another industrial standard, IEC
61508, named ``Functional safety of electrical/electronic/programmable
electronic safety-related systems'', which purpose is to ensure
functional safety in computer based systems' overall life cycles.
\cite{IEC61508}
\cite{ISO26262}

It is useful to distinguish between systems with different levels of
dependability, and determine where the hazards exists. When this risk analysis
is completed, and appropriate reliability and availability requirements is
assigned to the system, the system can be identified by a certain automotive
safety integrity level (ASIL).  If this number is high, the system will
experience a more rigorous design and testing than could be justified for a
lesser demanding system. These levels are more defined in the standards
IEC~61508 and ISO~26262.
\cite{COURSEBOOK:safety-critical}

%taget ifr�n wikipedia, b�ttre k�lla beh�vs. Dessv�rre r�cker
%det inte med v�ra egna slutsatser.
The ASIL can be said to be the measurement of functional safety. Hence a
product within the automotive industry can be claimed to be ''functional safe'' to a
certain ASIL.

\subsection{Testing}
Functional safety demands testing, and testing accounts for around half of all
software development costs . Reducing the cost is motivated and can be done by
automating the test generation process.
\cite{ISO26262}
\cite{QUICKCHECK:lightweight}
\cite{QUICKCHECK:software}
\cite{Testing:black_box}

%% TODO: skriv att det kallas state-space explosion problem i stycket nedan.
For simple devices it is possible to exhaustively test the functional
safety of the system. For example
consider a system consisting of a small number of switches, where each
switch has only two states; open and closed. Then the number of
possible failures can be determined by the combination of all possible
failure states of each individual switch. The complexity issue is that
in systems such as microprocessors or ECU's, the number of possible
failure states is so large that it is considered infinite. This makes
it impossible to exhaustively test the system, and therefore, make the
detection of failures unreliable.
\cite{COURSEBOOK:safety-critical}

\section{Purpose}
This thesis purpose is to automate the testing process of automotive
software in an effective and good way, and to make it possible to raise
the Automotive Safety Integrity Level (ASIL), where applicable.

It is desired to do an evaluation of tools that can be used in order
to perform at least semi formal verification of automotive software
modules. The main purpose is however to evaluate if Quviq QuickCheck
can be used to fulfill this.


\section{Objective}
The first problem is to evaluate what ``semi formal verification''
according to the ISO-standard means. In formal methods of mathematics,
formal verification means to prove the correctness of algorithms.  The
ISO-standard mentions both ``formal verification'' and ``semi formal
verification'' for software development, but it does not describe how
to realize any of these.  This evaluation must be performed to obtain
knowledge of how to properly implement functional safety and reach an
ASIL classification, using automated testing.

A model for an AUTOSAR module needs to be implemented. For this to be
a good model, some questions must first be answered.  How can one
achieve good test cases for the model? How can one tweak the test
generation to find test cases that are interesting in a safety
critical point of view? Is the implemented model together with the
generated tests good enough to reach the goals?

The test generation is a big problem when verifying a model. With unit
tests, one can argue that each line of code has been executed (100\%
code coverage), but that is just a statement for that everything has
been executed. Have it been executed correctly?  Are every combination
of computations in the system necessary to ensure correctness, or with
other words, is it possible to collapse some states in the system's
state space without endangering the safety of the whole system?

There must be an evaluation of the solution after the model has been
implemented.  Does it ensure functional safety? How can one measure
the size of the state space that is actually verified?  Even if test
generation is implemented properly, the solution might not be within
the ISO-standards means of functional safety.

%% TODO: Ska vi ha med konfidensintervall eller det bes�kta state spacet?
One must propose and motivate what should be done to be able to
achieve a semi formal verification. This can include a confidence
interval for how certain the verification is. The confidence interval
would help describing the visited state space because it is probably
not feasible to exhaustively visit all states due to the state space
explosion problem.

%% TODO: Skriv om sista meningen.
The main objective is to prove that it is possible to do semi formal
verification for an AUTOSAR module and its specification.  It should
not matter which configuration that is used or how the module is
implemented, because the specification should hold for all
configurations and implementations. Every company that implements the
module should be able to run the final code to achieve ``semi formal
verification''. Since modules in AUTOSAR are dependent, the work
presented here should be generalized so it can be hooked on when
implementing test suits, using the same techniques, for other modules.


\section{Scope}
%% TODO: �verv�g "will be" i meningen om att vi v�ljer Watchdog managern.
We will use AUTOSAR 4.0 revision 3 for our thesis work.  Since this
version of AUTOSAR consists of more than 100 specifications and other
auxiliary materials\cite{AUTOSAR:URL}, we will limit our scope to one
specification. The module of this thesis is the Watchdog Manager. This
module provides monitoring services used to maintain correctness. The
module is chosen because it got dependencies, and is used to report
development and production errors, but mainly because it executes
safety-critical work. The fact that it got dependencies is important
when doing integration testing between different modules.

The aim of the work is to verify software components. In other words
no work considered hardware or a combination of hardware and software
will be prioritized. All implemented code for the verification will
run on a standard PC-machine.

%% TODO: skriv n�got om varf�r vi inte vill testa segmentation fault configs?
We will not implement deprecated API-functions in our model, nor will
we test configurations which will give raise to segmentation faults.
