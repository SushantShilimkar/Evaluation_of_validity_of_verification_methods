It is possible to achieve some functional safety using QuickCheck, at
least within the software units. There are however a number of
ISO~26262 requirements that is not possible to achieve with only a
software testing tool. For example requirements that verifies the
hardware specifications or how the safety plan should be made. It is
hard to use the ISO~26262 V-model if the software units do
not follow system properties that has been verified by a system
model. Because AUTOSAR is written with informal syntax, it can't be
used to verify the software units. This means one must translate the
AUTOSAR requirements to formal syntax and verify that the formal
requirements means the same as the informal requirements.

It is important to not only measure the state space, but also the code
coverage. This can also be done with the use of QuickCheck. It is easy
to measure Erlang coverages, and it is also easy to specify which
compiler QuickCheck should use to perform C-code coverages.
QuickCheck gathers information about the state space, which is
outputted after a test has been run.

Both negative and positive testing can be implemented with the use of
QuickCheck. Negative testing can be time-consuming because the program
quickly comes to an absorbing state. It is therefor important to tweak
the generators correctly.

Integration tests can also easily be implemented, by connecting
several AUTOSAR modules. When doing this even more ISO~26262
requirements can be evaluated and verified.

One needs to be aware of the configuration of the AUTOSAR module which
is going to be implemented, because it may contain variables that is
not safe to turn off. It may also be difficult to reach the whole
state space if the configuration is to simple or to complex.