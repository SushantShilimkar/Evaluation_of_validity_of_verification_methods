\documentclass[11pt,a4paper]{article}
\usepackage[latin1]{inputenc}
\usepackage[T1]{fontenc}
\usepackage[english]{babel}

\begin{document}

\section{Requirements that is broken within the C code}
\begin{description}
  \item[WDGM286] \parbox[t]{0.8\linewidth}{WdgM\_DeInit should set the global
      status to WDGM\_GLOBAL\_STATUS\_DEACTIVATED if the global status was
    WDGM\_GLOBAL\_STATUS\_OK}
  \item[WDGM215] \parbox[t]{0.8\linewidth}{if global status was
      WDGM\_GLOBAL\_STATUS\_OK and becomes WDGM\_GLOBAL\_STATUS\_EXPIRED, don't
      increment WdgM\_ExpiredSupervisionCycles}
  \item[WDGM252] \parbox[t]{0.8\linewidth}{if CP belongs to a graph, that
      graph's activity flag is set to false, and the CP is not the initial CP of
      that graph, then logical supervision status for that SE shall be set to
      WDGM\_INCORRECT}
  \item[WDGM274] \parbox[t]{0.8\linewidth}{Same as [WDGM252] but for internal
      supervision}
  \item[WDGM219] \parbox[t]{0.8\linewidth}{if global status was
      WDGM\_GLOBAL\_STATUS\_EXPIRED, it exists a SE that has the local status
      WDGM\_LOCAL\_STATUS\_EXPIRED, ExpiredSupervisionCycles is less than or
      equal to the expired supervision cycle tolerance, then keep the global
      status EXPIRED and increment ExpiredSupervisionCycles}
  \item[WDGM216] \parbox[t]{0.8\linewidth}{if global status was
      WDGM\_GLOBAL\_STATUS\_OK, it exists a SE that has the local status
      WDGM\_LOCAL\_STATUS\_EXPIRED, ExpiredSupervisionCyclesTol is equal to
      zero, then the global status should be set to
      WDGM\_GLOBAL\_STATUS\_STOPPED}
  \item[WDGM182] \parbox[t]{0.8\linewidth}{if global status was WDGM\_GLOBAL\_STATUS\_OK or
      WDGM\_GLOBAL\_STATUS\_FAILED, for each SE that is activated in the new mode,
      retain the state of each local SE.}
  \item[p. 27] \parbox[t]{0.8\linewidth}{At Watchdog Manager initialization, all
      the Results of alive/deadline/logical supervision are set to correct.}
\end{description}

\section{Requirements that is broken within AUTOSAR}
\begin{description}
  \item[ambiguous meaning] the definition of what (deadline) timestamp is and were it shall
    be set is ambiguous. Is it cycles, ticks, seconds or microseconds? and where
    should it increment/set the timestamp?
  \item[conflicting requirements] [WDGM317], [WDGM139] because [WDGM186]
  \item[no requirement, p. 125] reset of alive counter in mainfunction
  \item[ambiguous meaning] return value not specified but E\_NOT\_OK seems logical
\end{description}

\section{Bugfixes and changes to the C code}
\begin{description}
  \item[WdgM\_Lcfg.c:232] \parbox[t]{0.8\linewidth}{commented a function call to other module (//.Switch\_currentMode((Rte\_ModeType\_WdgMMode)status);)}
  \item[WdgM\_Lcfg.c:170] \parbox[t]{0.8\linewidth}{commented a function call to other module
(//.Switch\_currentMode((Rte\_ModeType\_WdgMMode)status);)}
  \item[WdgM.c:2056] bugfix because [WDGM286]
  \item[WdgM.c:976] bugfix because [WDGM215]
  \item[WdgM.c:1836] bugfix because [WDGM252]
  \item[WdgM.c:989] bugfix because [WDGM219]
  \item[WdgM.c:989] bugfix because [WDGM216]
  \item[WdgM.c:1779] bugfix because [WDGM274]
  \item[WdgM.c:1386] bugfix because [WDGM182]
  \item[WdgM\_Lcfg.c:223] \parbox[t]{0.8\linewidth}{bugfix, added initialization
      to alive/deadline/logical supervision results to WDGM\_CORRECT (copied the
      if statement from row 199), because [p. 27]}
\end{description}

\end{document}
